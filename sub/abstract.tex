\maketitle  % command to print the title page with above variables
\setcounter{page}{1}
%---------------------------------------------------------------------
%                  영문 초록을 입력하시오
%---------------------------------------------------------------------
\begin{abstracts}     %this creates the heading for the abstract page
	\addcontentsline{toc}{section}{Abstract}  %%% TOC에 표시
	\noindent{
		In this study, the level 3 data were calculated by calculating the temporal and spatial average values ​​using the NOAA/AVHRR sea surface temperature (SST) level 2 data. The daily average value, weekly average value, and monthly average value of SST around the Korean Peninsula were calculated and displayed on the map. The data were distributed by the Korea Oceanic and Atmospheric Satellite Center (KOSC).
		This study needs to be compared with the in-situ sea temperature in the future, and it will serve as a basis for research to increase the reliability of SST observed from satellites in the long term.
	}
\end{abstracts}

%---------------------------------------------------------------------
%                  국문 초록을 입력하시오
%---------------------------------------------------------------------
\begin{abstractskor}        %this creates the heading for the abstract page
	\addcontentsline{toc}{section}{초록}  %%% TOC에 표시
	\noindent{
		본 연구에서는 NOAA/AVHRR의 해수면온도(SST) 레벨 2 자료를 이용하여 시간적, 공간적으로 평균값을 산출하여 레벨 3 자료를 산출하였다. 한반도 주변 해역에 대하여 SST의 일평균값, 주평균값, 월평균값을 산출하여 지도 위에 표출하였다. 본 연구에서 사용 한 자료는 해양위성센터(KOSC)에서 배포하는 것으로 그 통계값을 구하였다. 본 연구는 향후 실측한 해수면 온도와의 비교하는 연구가 필요하며, 장기적으로 인공위성에서 관측한 SST의 자료 신뢰도를 높이는 연구의 기초가 될 것이다.
	}
\end{abstractskor}


%----------------------------------------------
%   Table of Contents (자동 작성됨)
%----------------------------------------------
\cleardoublepage
\addcontentsline{toc}{section}{Contents}
\setcounter{secnumdepth}{3} % organisational level that receives a numbers
\setcounter{tocdepth}{3}    % print table of contents for level 3
\baselineskip=2.2em
\tableofcontents


%----------------------------------------------
%     List of Figures/Tables (자동 작성됨)
%----------------------------------------------
\cleardoublepage
\clearpage
\listoftables
% 표 목록과 캡션을 출력한다. 만약 논문에 표가 없다면 이 위 줄의 맨 앞에 
% `%' 기호를 넣어서 주석 처리한다.

\cleardoublepage
\clearpage
\listoffigures
% 그림 목록과 캡션을 출력한다. 만약 논문에 그림이 없다면 이 위 줄의 맨 앞에 
% `%' 기호를 넣어서 주석 처리한다.

\cleardoublepage
\clearpage
\renewcommand{\thepage}{\arabic{page}}
\setcounter{page}{1}
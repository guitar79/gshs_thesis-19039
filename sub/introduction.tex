\section{서론}

\subsection{연구의 필요성 및 목적}

복잡하게 구성된 지구의 순환 체계에서 Sea Surface Temperature (SST)는 빠뜨릴 수 없는 요소이다. 기후에 밀접하게 영향을 주고받는 SST는 몇몇 해역에서 대기에 강제력을 행사하고, 다른 해역에서는 대기에 영향을 받으며 억지력으로서 작용한다. 계절에 따라 SST와 대기가 미치는 영향의 비중이 달라지는 해역도 존재한다 \cite{wu2007regimes}.
 
SST는  태풍이나 집중호우 등의 위험기상의 발생가능성 또한 SST의 변동성과 연관지어 예측할 수 있는 만큼 SST를 관측하고 그 경향성을 파악하는 것은 지구 환경을 이해하는 데에 굉장히 중요하다 \cite{정은실2019한반도에서}. 

SST를 관측하는 방법으로는 크게 해양 부이를 이용한 관측과 인공위성 자료를 통한 산출법이 있다. 전자의 경우 구름과 같은 오차 원인을 배제하고 직접적으로 정확한 데이터를 얻을 수 있다는 장점이 있으나, 부이가 위치하는 한 점의 값만을 얻을 수 있기 때문에 폭넓은 지역의 해수면 온도를 알 수 없다는 단점이 있다. 그와는 반대로 인공위성 자료를 통한 산출법은 대기와 다른 여러 요인들로 인한 오차를 계산해야 하나, 위성으로 관측할 수 있는 광범위한 해역의 정보를 알 수 있다는 것이 장점이다. 

본 연구에서는 인공위성 자료를 이용하여 한반도 주변 해역의 SST를 산출해 보고자 한다.

\newpage
\subsection{이론적 배경}

\subsubsection{기상 위성}

기상위성이란 지구의 기상현상과 대기를 관측하기 위한 목적의 인공위성들의 분류이며, 우리가 현재 사용하는 기상위성은 궤도에 따라 정지궤도위성과 극궤도위성으로 나뉜다. 

정지궤도위성은 적도 상공에 위치해, 약 $35,800 \textrm{ km}$ 높이에서 지구와 같은 각속도로 지구 주위를 공전하기 때문에 지상의 관측자가 보았을 때에는 하늘에 고정된 것처럼 느껴지므로 이와 같은 명칭이 붙었다. 정지궤도위성은 지구의 약 $\frac{1}{4}$ 정도 되는 고정된 면적을 관측할 수 있으며 이 때문에 한 지역의 연속적인 기상 상태 변화 등을 관찰하는 데에 있어 유용하다. 

극궤도위성은 남극과 북극을 통과하여 지구 주위를 공전하는 위성으로, 고도는 약 $800 - 1,500 \textrm{ km}$ 정도이다. 이는 하루에 전체 지구를 약 2회 관측할 수 있으며, 고도가 기상위성에 비해 낮아 세기가 약한 파장도 인식할 수 있으며, 극지의 얼음, 해양, 에너지의 순환 등 다양한 현상을 관측할 수 있다.


\subsubsection{NOAA 위성}

National Oceanic and Atmospheric Administration (NOAA)에서 진행하는 Polar Operational Environmental Satellite (POES) 프로젝트의 일부로 NOAA 위성을 운용하고 있다. 이 위성은 직하점을 중심으로 $55.4 {^\circ}$ 안쪽의 범위를 주사할 수 있다. 탑재되어 있는 주 관측 센서는 Advanced Very High Resolution Radiometer (AVHRR)와 Television InfraRed Observation Satellite Operational Vertical Sounder (TOVS) 등이 있다. 이 가운데 AVHRR은 5개의 채널을 가졌으며 각각의 파장과 주 용도는 Table \ref{table:AVHRR}과 같다.

% Please add the following required packages to your document preamble:
% \usepackage{multirow}
\begin{table}[!htbp]
	\caption{Description for AVHRR channels Channel.}
	\begin{center}

		\begin{tabular}{c|c|c}
			\toprule
			Channel Number & \begin{tabular}[c]{@{}c@{}}Wavelength\\ {($\rm{\mu m}$)}\end{tabular} & Typical Use                                                                           \\ 		\toprule
			1              & 0.58 $\sim$0.68                                          & Daytime cloud and surface mapping                                                     \\ \hline
			2              & 0.725 $\sim$1.00                                         & Land-water boundaries                                                                 \\ \hline
			3a             & 1.58 $\sim$1.64                                          & Snow and ice detection                                                                \\ \hline
			3b             & 3.55 $\sim$3.93                                          & \begin{tabular}[c]{@{}l@{}}Night cloud mapping, Sea surface temperature\end{tabular} \\ \hline
			4              & 10.30 $\sim$11.30                                        & \begin{tabular}[c]{@{}l@{}}Night cloud mapping, Sea surface temperature\end{tabular} \\ \hline
			5              & 11.50 $\sim$12.50                                        & Sea surface temperature                                                               \\ 			\bottomrule
		\end{tabular}

	\end{center}
	\label{table:AVHRR}
\end{table}


\subsubsection{Terra/Aqua 위성}

1999년 12월 18일 발사되어 2000년 2월 24일 부터 자료를 송신한 Terra (EOS AM-1) 위성은 하루에 한 지점을 2번 관측하는 극궤도위성이다. 지구 환경과 기후의 변화를 관측하는 것이 목표인 이 위성은 Advanced Spaceborne Thermal Emission and Reflection Radiometer (ASTER), Clouds and the Earth's Radiant Energy System (CERES), Multi-angle Imaging SpectroRadiometer (MISR), Moderate-resolution Imaging Spectroradiometer (MODIS),  Measurements of Pollution in the Troposphere (MOPITT) 로 총 6 가지의 센서들을 탑재하였다. 

Aqua 위성은 2002년 5월 4일 지표면과 대기 중의 물에 관한 연구를 위하여 발사되었으며, Atmospheric Infrared Sounder (AIRS), the Advanced Microwave Sounding Unit (AMSU-A), the Humidity Sounder for Brazil (HSB), the Advanced Microwave Scanning Radiometer for EOS (AMSR-E), the Moderate-Resolution Imaging Spectroradiometer (MODIS), and the Clouds and the Earth's Radiant Energy System (CERES)로 총 6가지 센서들을 탑재하였으나, 그중 AMSR-E와 HSB가 손상되어 작동을 멈추었고, AMSU-A와 CERES는 일부 고장이 발생하였으나 여전히 작동하고 있다. Terra와 Aqua 위성은 Aura 위성과 함께 Earth Observing System(EOS)의 일부이다. 

MODIS는 Terra와 Aqua 위성의 핵심 탑재체이다. 크기 $1.0\textrm{ m} \times 1.6 \textrm{ m} \times 1.0 \textrm{ m}$, 질량 $228.7 \textrm{ kg}$의 MODIS는 위성에 탑재되어 $705 \textrm{ km}$의 고도에서 $55 \circ$의 시야각, $2330 \textrm{ km}$의 관측폭으로 하루 한 번 혹은 두 번 같은 지점을 관측한다. 총 36 개인 각 채널의 해상도는 각각 $250 \textrm{ m}$(채널 1 - 2), $500 \textrm{ m}$(채널 3 - 7), $1 \textrm{ km}$(채널 8 - 36)이며 그 중 SST 관측에 쓰이는 것은 약 $3.7 - 4.1 \textrm{ }\rm{\mu m}$의 대역폭을 가지고 있는 20, 21, 22, 23 번 채널과 $10.8 - 12.3 \textrm{ }\rm{\mu m}$의 31, 32 번 채널이다. 자세한 정보는 Table \ref{table:MODIS}에 나타내었다.


% Please add the following required packages to your document preamble:
% \usepackage{multirow}
\begin{table}[!htbp]
	\caption{Description for MODIS channels.}
	\begin{tabular}{l|c|c|c|c}
		\toprule
		Primary Use                                                                                              & Band & Bandwidth1  & \begin{tabular}[c]{@{}l@{}}Spectral\\ Radiance\end{tabular} & Required SNR \\ 
		\toprule
		
		\multirow{2}{*}{\begin{tabular}[c]{@{}l@{}}Land/Cloud/Aerosols\\ Boundaries\end{tabular}}                & 1    & 620 - 670   & 21.8                                                         & 128           \\ \cline{2-5} 
		& 2    & 841 - 876   & 24.7                                                         & 201           \\ \hline
		\multirow{5}{*}{\begin{tabular}[c]{@{}l@{}}Land/Cloud/Aerosols\\ Properties\end{tabular}}                & 3    & 459 - 479   & 35.3                                                         & 243           \\ \cline{2-5} 
		& 4    & 545 - 565   & 29.0                                                         & 228           \\ \cline{2-5} 
		& 5    & 1230 - 1250 & 5.4                                                          & 74            \\ \cline{2-5} 
		& 6    & 1628 - 1652 & 7.3                                                          & 275           \\ \cline{2-5} 
		& 7    & 2105 - 2155 & 1.0                                                          & 110           \\ \hline
		\multirow{9}{*}{\begin{tabular}[c]{@{}l@{}}Ocean Color/\\ Phytoplankton/\\ Biogeochemistry\end{tabular}} & 8    & 405 - 420   & 44.9                                                         & 880           \\ \cline{2-5} 
		& 9    & 438 - 448   & 41.9                                                         & 838           \\ \cline{2-5} 
		& 10   & 483 - 493   & 32.1                                                         & 802           \\ \cline{2-5} 
		& 11   & 526 - 536   & 27.9                                                         & 754           \\ \cline{2-5} 
		& 12   & 546 - 556   & 21.0                                                         & 750           \\ \cline{2-5} 
		& 13   & 662 - 672   & 9.5                                                          & 910           \\ \cline{2-5} 
		& 14   & 673 - 683   & 8.7                                                          & 1087          \\ \cline{2-5} 
		& 15   & 743 - 753   & 10.2                                                         & 586           \\ \cline{2-5} 
		& 16   & 862 - 877   & 6.2                                                          & 516           \\ \hline
		\multirow{3}{*}{\begin{tabular}[c]{@{}l@{}}Atmospheric\\ Water Vapor\end{tabular}}                       & 17   & 890 - 920   & 10.0                                                         & 167           \\ \cline{2-5} 
		& 18   & 931 - 941   & 3.6                                                          & 57            \\ \cline{2-5} 
		& 19   & 915 - 965   & 15.0                                                         & 250           \\ 
		
		\toprule

		\multirow{4}{*}{\begin{tabular}[c]{@{}l@{}}Surface/Cloud\\ Temperature\end{tabular}} & 20   & 3.660 - 3.840   & 0.45(300K)                                                   & 0.05                                                              \\ \cline{2-5} 
		& 21   & 3.929 - 3.989   & 2.38(335K)                                                   & 0.20                                                              \\ \cline{2-5} 
		& 22   & 3.929 - 3.989   & 0.67(300K)                                                   & 0.07                                                              \\ \cline{2-5} 
		& 23   & 4.020 - 4.080   & 0.79(300K)                                                   & 0.07                                                              \\ \hline
		\multirow{2}{*}{\begin{tabular}[c]{@{}l@{}}Atmospheric\\ Temperature\end{tabular}}   & 24   & 4.433 - 4.498   & 0.17(250K)                                                   & 0.25                                                              \\ \cline{2-5} 
		& 25   & 4.482 - 4.549   & 0.59(275K)                                                   & 0.25                                                              \\ \hline
		\multirow{3}{*}{\begin{tabular}[c]{@{}l@{}}Cirrus Clouds\\ Water Vapor\end{tabular}} & 26   & 1.360 - 1.390   & 6.00                                                         & 150(SNR)                                                          \\ \cline{2-5} 
		& 27   & 6.535 - 6.895   & 1.16(240K)                                                   & 0.25                                                              \\ \cline{2-5} 
		& 28   & 7.175 - 7.475   & 2.18(250K)                                                   & 0.25                                                              \\ \hline
		Cloud Properties                                                                     & 29   & 8.400 - 8.700   & 9.58(300K)                                                   & 0.05                                                              \\ \hline
		Ozone                                                                                & 30   & 9.580 - 9.880   & 3.69(250K)                                                   & 0.25                                                              \\ \hline
		\multirow{2}{*}{\begin{tabular}[c]{@{}l@{}}Surface/Cloud\\ Temperature\end{tabular}} & 31   & 10.780 - 11.280 & 9.55(300K)                                                   & 0.05                                                              \\ \cline{2-5} 
		& 32   & 11.770 - 12.270 & 8.94(300K)                                                   & 0.05                                                              \\ \hline
		\multirow{4}{*}{\begin{tabular}[c]{@{}l@{}}Cloud Top\\ Altitude\end{tabular}}        & 33   & 13.185 - 13.485 & 4.52(260K)                                                   & 0.25                                                              \\ \cline{2-5} 
		& 34   & 13.485 - 13.785 & 3.76(250K)                                                   & 0.25                                                              \\ \cline{2-5} 
		& 35   & 13.785 - 14.085 & 3.11(240K)                                                   & 0.25                                                              \\ \cline{2-5} 
		& 36   & 14.085 - 14.385 & 2.08(220K)                                                   & 0.35                                                              \\ 		\bottomrule
		
	\end{tabular}
	\label{table:MODIS}
\end{table}


\subsubsection{인공위성 자료}

인공위성 자료는 처리 정도에 따라 레벨 0, 레벨 1A, 레벨 1B, 레벨 2, 레벨 3, 레벨 4 데이터로 나뉜다\cite{Level}. 

레벨 0 데이터는 우주선에서 지상으로 전송하는 데 쓰이는 통신 정보만을 제거한 상태의 페이로드 데이터를 의미하며, 레벨 1A 데이터는 시간을 참조하여 레벨 0 데이터를 재구성하고 기하적 보정 등 보조 자료를 주석으로 추가한 상태이다. 레벨 1B 데이터는 그것에서 센서의 특성과 복사량에 대한 보정이 이루어진 결과물로, 이 단계부터는 센서 보정이 변경된다면 다른 데이터로 대체되어야만 한다. 

레벨 2 데이터는 이들을 이용하여 지구물리학적으로 의미있는 변수들을 도출하여 SST(Sea Surface Temperature), OC(Ocean Color) 등의 그룹으로 분류한 것이고, 레벨 3 데이터는 그러한 데이터를 일정 기간 동안 일정 구역 집계한 기록이다. 

마지막으로 레벨 4 데이터는 하위 레벨 데이터에 대한 분석을 말한다. 

본 연구에서는 인공위성을 이용한 SST 산출 방식을 채택하여 NOAA 위성의 AVHRR 센서로 관측한 레벨 2 데이터를 레벨 3 데이터로 가공하여 분석하는 것이 목적이다. 

\subsubsection{SST 산출 알고리즘}

인공위성 자료를 통해 SST 데이터를 산출하는 데에는 MCSST(Multi-Channel Sea Surface Temperature)와 CPSST(Cross Product Sea Surface Temperature) 등 여러 기법이 존재한다. (박경혜, 정종률, 최병호, 김구, 1994) SST 산출에 쓰이는 채널은 22, 23번(단파)와 31, 32번(장파)이며, 각각의 채널에서는 지표면을 흑체로 가정하고 슈테판-볼츠만 법칙을 이용하여 밝기온도를 구한다. McMillin과 Crosby(1984)의 연구 결과에 의하면 수증기흡수계수 ki, kj에 대하여 =kjkj-ki일 때, SST=Tj+(Ti-Tj)의 값을 가진다. 

그렇게 도출한 단일채널 SST의 값을 이용하여 아래와 같은 총 세 가지 기법으로 MCSST를 산출한다 \cite{walton1988nonlinear}. 

MCSST(3, 4)=T11+1.616(T3.7-T11)+1.07 (dual window)
MCSST(4, 5)=T12+3.15(T11-T12)+0.10 (split window)
MCSST(3, 4, 5)=T11+0.943(T3.7-T12)+0.61 (triple window)

MCSST를 구하는 식에서는 수증기의 적외선 흡수율이 상수라고 가정하나, 실제로는 온도와 관계 있는 비선형적 함수로서 나타나고, 이에 따라 건조한 극지방이나 고온의 지역에서 산출한 결과와는 오차가 발생하게 된다. 따라서 이를 보완하기 위하여 개발된 비선형 알고리즘이 CPSST이다. (Walton et al, 1998)

\section{서론}

\subsection{연구의 필요성 및 목적}

복잡하게 구성된 지구의 순환 체계에서 해수면 온도(Sea Surface Temperature, SST)는 빠뜨릴 수 없는 요소이다. 기후에 밀접하게 영향을 주고받는 SST는 몇몇 해역에서 대기에 강제력을 행사하고, 다른 해역에서는 대기에 영향을 받으며 억지력으로서 작용한다. 계절에 따라 해수면온도와 대기가 미치는 영향의 비중이 달라지는 해역도 존재한다. (Wu, R., Kirtman, B.P. Regimes of seasonal air–sea interaction and implications for performance of forced simulations. Clim Dyn 29, 393–410 (2007).) SST는  태풍이나 집중호우 등의 위험기상의 발생가능성 또한 해수면온도의 변동성과 연관지어 예측할 수 있는 만큼(정은실. (2019). 한반도에서 위험기상 발생 시 나타나는 해수면온도 변동의 특성. 한국지구과학회지, 40(3), 240-258.) SST를 관측하고 그 경향성을 파악하는 것은 지구 환경을 이해하는 데에 굉장히 중요하다. 수온을 관측하는 방법으로는 크게 해양 부이를 이용한 관측과 인공위성 자료를 통한 산출법이 있다. 전자의 경우 구름과 같은 오차원인을 배제하고 직접적으로 정확한 데이터를 얻을 수 있다는 장점이 있으나, 부이가 위치하는 한 점의 값만을 얻을 수 있기 때문에 폭넓은 지역의 해수면 온도를 알 수 없다는 단점이 있다. 그와는 반대로 인공위성 자료를 통한 산출법은 대기와 다른 여러 요인들로 인한 오차를 계산해야 하나, 위성으로 관측할 수 있는 광범위한 해역의 정보를 알 수 있다는 것이 장점이다. 해수면 온도 데이터는 처리 정도에 따라 레벨 0, 레벨 1A, 레벨 1B, 레벨 2, 레벨 3, 레벨 4 데이터로 나뉜다. 레벨 0 데이터는 우주선에서 지상으로 전송하는 데 쓰이는 통신 정보만을 제거한 상태의 페이로드 데이터를 의미하며, 레벨 1A 데이터는 시간을 참조하여 레벨 0 데이터를 재구성하고 기하적 보정 등 보조 자료를 주석으로 추가한 상태이다. 레벨 1B 데이터는 그것에서 센서의 특성과 복사량에 대한 보정이 이루어진 결과물로, 이 단계부터는 센서 보정이 변경된다면 다른 데이터로 대체되어야만 한다. 레벨 2 데이터는 이들을 이용하여 지구물리학적으로 의미있는 변수들을 도출하여 SST(Sea Surface Temperature), OC(Ocean Color) 등의 그룹으로 분류한 것이고, 레벨 3 데이터는 그러한 데이터를 일정 기간 동안 일정 구역 집계한 기록이다. 마지막으로 레벨 4 데이터는 하위 레벨 데이터에 대한 분석을 말한다. (Ocean Color Web July 23, 2021) 본 연구에서는 인공위성을 이용한 SST 산출 방식을 채택하여 NOAA 위성의 AVHRR 센서로 관측한 레벨 2 데이터를 레벨 3 데이터로 가공하여 분석하는 것이 목적이다. 선행연구로는 Terra 위성의 MODIS 센서를 이용한 SST (Sea Surface Temperature) 산출 연구 (정주용 외, 2002), 구름 제거 기법과 구름 영향에 따른 신뢰도 부여에 관한 연구 (양성수, 양찬수, 박광순, 2010), TeraScan 시스템에서 NOAA/AVHRR 해수면 온도 산출시 구름 영향에 따른 신뢰도 부여 기법에 관한 연구 등이 있다.


\subsection{이론적 배경}

\subsubsection{기상 위성}

기상위성이란 지구의 기상현상과 대기를 관측하기 위한 목적의 인공위성들의 분류이며, 우리가 현재 사용하는 기상위성은 궤도에 따라 정지궤도위성과 극궤도위성으로 나뉜다. 정지궤도위성은 적도 상공에 위치해, 약 $35,800 ~\rm{km}$ 높이에서 지구와 같은 각속도로 지구 주위를 공전하기 때문에 지상의 관측자가 보았을 때에는 하늘에 고정된 것처럼 느껴지므로 이와 같은 명칭이 붙었다. 정지궤도위성은 지구의 ¼ 정도 되는 고정된 면적을 관측할 수 있으며 이 때문에 한 지역의 연속적인 기상 상태 변화 등을 관찰하는 데에 있어 유용하다. 극궤도위성은 남극과 북극을 통과하여 지구 주위를 공전하는 위성으로, 고도는 약 800 ~ 1,500 km의 궤도를 갖는다. 이는 100분 마다 지구를 한바퀴 공전하여 하루에 전체 지구를 약 2회 관측할 수 있으며, 고도가 기상위성에 비해 낮아 세기가 약한 파장도 인식할 수 있으며, 극지의 얼음, 해양, 에너지의 순환 등 다양한 현상을 관측할 수 있다. (국가기상위성센터 2021년 7월 23일)

\subsubsection{NOAA (National Oceanic and Atmospheric Administration) 위성}

미해양대기청(NOAA, National Oceanic and Atmospheric Administration)에서 진행하는 POES(Polar Operational Environmental Satellite)프로젝트의 일부로 NOAA 위성을 운용하고 있다. 하는 , 직하점을 중심으로 55.4°안쪽의 범위를 주사할 수 있다. 탑재되어 있는 주 관측 센서는 AVHRR(Advanced Very High Resolution Radiometer)와 TOVS(Television InfraRed Observation Satellite Operational Vertical Sounder) 등이 있다. 이 가운데 AVHRR은 5개의 채널을 가졌으며 각각의 파장과 주 용도는 Table 3과 같다.

\subsubsection{Terra/Aqua 위성}

1999 년 12 월 18 일 발사되어 다음 년도 2 월 24 일부터 자료를 송신한 Terra (EOS AM-1) 위성은 하루에 한 지점을 2번 관측하는 극궤도위성이다. 지구 환경과 기후의 변화를 관측하는 것이 목표인 이 위성은 ASTER (Advanced Spaceborne Thermal Emission and Reflection Radiometer), CERES (Clouds and the Earth's Radiant Energy System), MISR (Multi-angle Imaging SpectroRadiometer), MODIS (Moderate-resolution Imaging Spectroradiometer), MOPITT (Measurements of Pollution in the Troposphere) 로 총 6 가지의 센서들을 탑재하였다. 

Aqua 위성은 2002 년 5 월 4 일 지표면과 대기 중의 물에 관한 연구를 위하여 발사되었으며, AMSR-E (Advanced Microwave Scanning Radiometer-EOS), MODIS (Moderate Resolution Imaging Spectroradiometer), AMSU-A (Advanced Microwave Sounding Unit), AIRS (Atmospheric Infrared Sounder), AIRS (Atmospheric Infrared Sounder), HSB (Humidity Sounder for Brazil), CERES (Clouds and the Earth's Radiant Energy System)로 총 6 가지 센서들을 탑재하였으나, 그중 AMSR-E와 HSB가 손상되어 작동을 멈추었고, AMSU-A와 CERES는 일부 고장이 발생하였으나 여전히 작동하고 있다. Terra와 Aqua 위성은 Aura 위성과 함께 EOS (Earth Observing System) 의 일부이다. 

MODIS(MODerate resolution Imaging Spectrometer)는 Terra와 Aqua 위성의 핵심 탑재체이다. 크기 1.0 m X 1.6 m X 1.0 m, 질량 228.7 kg의 MODIS는 위성에 탑재되어 705 km의 고도에서 55도의 시야각, 2330 km의 관측폭으로 하루 한 번 혹은 두 번 같은 지점을 관측한다. 총 36 개인 각 채널의 해상도는 각각 250 m(채널 1 ~ 2), 500 m(채널 3 ~ 7), 1 km(채널 8 ~ 36)이며 그 중 SST 관측에 쓰이는 것은 약 3.7 ~ 4.1 µm의 대역폭을 가지고 있는 20, 21, 22, 23 번 채널과 10.8 ~ 12.3 µm의 31, 32 번 채널이다.

\subsection{SST 산출 알고리즘}
\section{연구 방법 및 과정}

\subsection{데이터 파악 및 수집}

해양위성센터에서 제공하는 SST 데이터를 다운받을 수 있는 경로를 확인하였다. 접근할 수 있는 데이터는 2020년 4월 29일 기준으로 Table 과 같다.

2012년부터 2019년까지의 8년 동안 Terra/Aqua 위성이 MODIS를 통해 수집한 자료를 연구에 이용하기로 결정하고, Github에서 다운로드한 웹 크롤링 파일을 이용하여 해양위성센터의 SST 데이터를 크롤링하는 table 과 같이 코드를 작성하였다. 2021년 6월 현재는 천리안위성 2호가 서비스를 시작하면서 사이트가 개편되어 데이터 배포 방식이 바뀌었어 아래의 코드가 실행되지 않을 수도 있다.

온라인 원격수업 환경에서 파일을 다운로드받기 위해 Chrome Remote Desktop을 이용하여 개인 노트북을 Ubuntu 운영체제의 서버 컴퓨터와 연결하여 사용할 수 있도록 하였으며, Ubuntu 프롬프트 명령어를 사용하여 파일 디렉토리를 탐색하는 방법을 학습하였다. 오랜 시간 동안 많은 양의 데이터를 다운받아야 하기 때문에 도중에 프롬프트 창을 닫더라도 계속 다운받을 수 있도록 nohup 명령어를 이용하여 백그라운드로 파일을 실행하였다. 


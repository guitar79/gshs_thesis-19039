\section{연구 방법 및 과정}

\subsection{데이터 파악}

해양위성센터에서 제공하는 SST 데이터를 다운받을 수 있는 경로를 확인하였다. 접근할 수 있는 데이터는 2020년 4월 29일 기준으로 Table \ref{table:KOSC-SST-data}\와 같다.


\begin{table}[!htbp]
	\caption{해양위성센터에서 다운로드 가능한 SST 데이터.}
	\begin{tabular}{c|c|c}
		\toprule
		센서명           & 자료시작시기       & 자료종료시기 (2020. 4. 29. 기준) \\ \toprule
		AVHRR         & 2011. 9. 1.  & 2020. 4. 21.          \\ \hline
		MODIS (Aqua)  & 2011. 9. 1.  & 2020. 4. 6.           \\ \hline
		MODIS (Terra) & 2011. 9. 1.  & 2020. 4. 7.           \\ \hline
		VIIRS         & 2016. 6. 17. & 2020. 4. 27.          \\ \bottomrule
	\end{tabular}

	\label{table:KOSC-SST-data}
\end{table}

NOAA/AVHRR 자료는 Fig. \ref{fig:asc_file}\와 같이 텍스트 파일의 형태로 배포되고 있다는 것을 알 수 있었다. 총 4개의 열로 저장되어 있으며, 첫번째 열부터 각각 인덱스, 위도, 경도, SST 임을 알 수 있는데, 자료가 산출되지 않은 경우에 ***로 표시되어 있다.

\begin{figure}[htbp]
	\centerline{\includegraphics[width=12cm]{asc_file1}}
	\caption{NOAA/AVHRR SST 자료 텍스트 파일 캡처 화면.}
	\label{fig:asc_file}
\end{figure}

Terra/Aqua 위성의 MODIS를 구한 SST 자료는 HDF(Hierarchical Data Format) 형태로 배포되었다. HDF는 이름 그대로 계층적으로 구조화된 다차원 배열 데이터를 저장하기 위하여 HDF Gruop(https://www.hdfgroup.org/)에 의해 만들어진 파일 형식이다.

\newpage
\subsection{연구에 사용한 데이터}

해양위성센터에서 배포한 MODIS의 SST 데이터는 구름이 제거되지 않아서 SST 값에 심각한 오류를 포함하고 있어 사용하지 않고, NOAA/AVHRR의 SST 레벨2 자료를 이용하여 연구를 진행하였다. 

NOAA/AVHRR의 SST 레벨2 자료는 앞서 언급한 것 처럼 텍스트 파일 형태로 제공되고 있고, Figure \ref{fig:SST-KOSC}\와 같이 지도 위에 표출된 자료도 함께 제공되고 있다. 

\begin{figure}[htbp]
	\centerline{\includegraphics[width=10cm]{2019.0104.1034.noaa-18.sst}}
	\caption{KOSC에서 배포한 NOAA/AVHRR SST 자료.}
	\label{fig:SST-KOSC}
\end{figure}

KOSC로 부터 다운받아 본 연구에 사용한 NOAA/AVHRR의 SST 자료의 정보는 Table \ref{table:NOAA-data}\와 같다.

\begin{table}[!htbp]
	\caption{사용한 NOAA/AVHRR SST data.}

	\begin{tabular}{c|c|c|c|c|c}
		\hline
		
		\hline
		year   & NOAA-15 & NOAA-16 & NOAA-17 & NOAA-18 & NOAA-19 \\ 
		
		\hline
		
		\hline
		2011 & 0       & 406     & 18      & 412     & 413     \\ \hline
		2012 & 12      & 1,184   & 11      & 1,073   & 1,141   \\ \hline
		2013 & 0       & 1,229   & 0       & 1,085   & 1,145   \\ \hline
		2014 & 0       & 533     & 0       & 1,056   & 728     \\ \hline
		2015 & 0       & 0       & 0       & 1,106   & 452     \\ \hline
		2016 & 0       & 0       & 0       & 1,022   & 1,177   \\ \hline
		2017 & 0       & 0       & 0       & 818     & 1,072   \\ \hline
		2018 & 0       & 0       & 0       & 912     & 937     \\ \hline
		2019 & 0       & 0       & 0       & 847     & 843     \\ \hline
		2020 & 0       & 0       & 0       & 526     & 527     \\ 
		
		\hline

		\hline
	\end{tabular}
	\label{table:NOAA-data}
\end{table}


\newpage
\subsection{자료 처리}

NOAA/AVHRR의 SST 레벨2 자료를 위도, 경도 구간을 나눈 후, 일평균값, 주평균값, 월평균값을 산출하여 레벨3 자료를 만들었다. 자료 처리는 Phthon을 이용하여 실시하였다. 

\subsection{해역의 구분}

국립해양조사원에서는 부산 최남단의 팔각정을 지나는 $135 \rm{{^\circ} E}$ 선을 기준으로 동해와 남해를 구분하며, 전남 해남군 송지면 송호리 갈두산 사자봉 땅끝탑에서 그은 $225 \rm{{^\circ} E}$ 선을 기준으로 남해와 서해를 구분한다. 동해 해수면 온도에 대한 김진은, 차동현(2017)의 연구에서는 동해의 경계를 $35 {^\circ} \sim 39 \rm{{^\circ} N}$, $128 \rm{{^\circ}} \sim 135 \rm{{^\circ} E}$로 규정하고 있다 \cite{김진은2017영동}. 


